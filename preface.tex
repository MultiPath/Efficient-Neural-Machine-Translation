\prefacesection{Abstract}
The dream of automatic translation that builds the communication bridge between people from different civilizations dates back to thousands of years ago. 
For the past decades, researchers devoted to proposing practical plans, from rule-based machine translation to statistical machine translation. 
Till recent years, with the general success of artificial intelligence (AI) and the emergence of neural network models, a.k.a. Deep learning, neural machine translation (NMT), 
as the new generation of machine translation framework based on sequence-to-sequence learning has achieved the state-of-the-art and even human-level translation performance on a variety of languages.

The impressive achievements brought by NMT mainly owes to its deep neural network structures with massive amounts of parameters, 
which can be efficiently tuned from vast volume of parallel data in the order of tens or hundreds of millions of sentences. 
Unfortunately, in spite of the success neural systems also bring about new challenges to machine translation, in which one of the central problems is efficiency. 
The efficiency issue involves two aspects: 
(1) NMT is data-hungry because of its vast size of parameters, which makes training a reasonable model difficult in practice for low resource cases. 
For instance, most of the human languages do not have enough parallel data with other languages to learn an NMT model; 
Moreover, documents in specialized domains such as law or medical files usually contain tons of professional translations, leading to less efficient for NMT to learn from;
(2) NMT is slow for computation compared to conventional methods due to its deep structure and limitations of the decoding algorithms. 
Especially the low efficiency at inference time profoundly affects the real-life application and the smoothness of the communication. 
In some cases like video conference, we also hope the neural system to translate at real-time which, however, is difficult for the existing NMT models. 

This dissertation attempts to tackle these two challenges, respectively.
Contributions are twofold: 
(1) we address the data-efficiency challenges presented by existing NMT models and introduce insights based on the characteristics of the data, which includes 
(a) developing the copy-mechanism to target on rote memories in translation and general sequence-to-sequence learning; 
(b) using a non-parametric search-engine to guide the NMT system to perform well in special domains; 
(c) inventing a universal NMT system to extremely low resource languages; 
(d) extending the universal NMT system to be able to efficiently adapt to new languages by combing with meta-learning.
(2) for the decoding-efficiency challenges, we develop novel structures and learning algorithms 
(a) recasting the decoding of NMT in a trainable manner to achieve state-of-the-art performance with less time; 
(b) inventing the non-autoregressive NMT system which enables translation in entirely parallel; 
(c) developing the NMT model that learns to translate in real-time using reinforcement learning.

(449 words)


%This dissertation describes a new technique for learning open-domain knowledge from unstructured web-scale text corpora,
%  making use of a probabilistic relaxation of natural logic -- 
%  a logic which uses the syntax of natural language as its logical formalism.
%We begin by reviewing the theory behind natural logic, and propose a novel extension of the logic to handle
%  propositional formulae.
%
%We then show how to capture common sense facts: given a candidate statement about the world and a large corpus of 
%  known facts, is the statement likely to be true? 
%This is treated as a search problem 
%  from the query statement to its appropriate support in the knowledge base over valid (or approximately valid) 
%  natural logical inference steps.
%This approach achieves a 4x improvement at retrieval recall compared to lemmatized lookup, 
%  maintaining above 90\% precision.
%
%We then extend the approach to handle longer, more complex premises by segmenting these utterance into a set of 
%  atomic statements entailed through natural logic.
%We evaluate this system in isolation by using it as the main component in an Open Information Extraction system, 
%  and show that it achieves a 3\% absolute improvement in F1 compared to prior work on a competitive knowledge 
%  base population task.
%
%Finally, we address how to elegantly handle situations where we could not find a supporting premise for our query.
%To address this, we create an analogue of an evaluation function in gameplaying search: a shallow lexical 
%  classifier is folded into the search program to serve as a heuristic function to assess how likely we would 
%  have been to find a premise.
%Results on answering 4th grade science questions show that this method improves over both the classifier in isolation, 
%  a strong IR baseline, and prior work.
